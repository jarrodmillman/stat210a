\documentclass[twoside]{article}
\usepackage{amsmath,amssymb,amsthm,graphicx}
\usepackage{epsfig}
\usepackage[authoryear]{natbib}

\newcommand{\reals}{\mathbf{R}}
\newcommand{\integers}{\mathbf{Z}}
\newcommand{\naturals}{\mathbf{N}}
\newcommand{\rationals}{\mathbf{Q}}

% Caligraphic alphabet
\newcommand{\calr}{\mathcal{R}} % only because \cr already taken
\newcommand{\ca}{\mathcal{A}} \newcommand{\cb}{\mathcal{B}} \newcommand{\cc}{\mathcal{C}} \newcommand{\cd}{\mathcal{D}} \newcommand{\ce}{\mathcal{E}} \newcommand{\cf}{\mathcal{F}} \newcommand{\cg}{\mathcal{G}} \newcommand{\ch}{\mathcal{H}} \newcommand{\ci}{\mathcal{I}} \newcommand{\cj}{\mathcal{J}} \newcommand{\ck}{\mathcal{K}} \newcommand{\cl}{\mathcal{L}} \newcommand{\cm}{\mathcal{M}} \newcommand{\cn}{\mathcal{N}} \newcommand{\co}{\mathcal{O}} \newcommand{\cp}{\mathcal{P}} \newcommand{\cq}{\mathcal{Q}} \newcommand{\cs}{\mathcal{S}} \newcommand{\ct}{\mathcal{T}} \newcommand{\cu}{\mathcal{U}} \newcommand{\cv}{\mathcal{V}} \newcommand{\cw}{\mathcal{W}} \newcommand{\cx}{\mathcal{X}} \newcommand{\cy}{\mathcal{Y}} \newcommand{\cz}{\mathcal{Z}}

\newcommand{\ind}[1]{1_{#1}} % Indicator function
\newcommand{\pr}{P} % Generic probability
\newcommand{\ex}{E} % Generic expectation
\newcommand{\var}{\textrm{Var}}
\newcommand{\cov}{\textrm{Cov}}
\newcommand{\sgn}{\textrm{sgn}}
\newcommand{\sign}{\textrm{sign}}
\newcommand{\kl}{\textrm{KL}} 

\newcommand{\law}{\mathcal{L}}  % \law{X}, the measure associated with r.v. X
\newcommand{\normal}{N} % for normal distribution (can probably skip this)
\newcommand{\eps}{\varepsilon}

% Convergence
\newcommand{\convd}{\stackrel{d}{\longrightarrow}} % convergence in distribution/law/measure
\newcommand{\convp}{\stackrel{P}{\longrightarrow}} % convergence in probability
\newcommand{\convas}{\stackrel{\textrm{a.s.}}{\longrightarrow}} % convergence almost surely

\newcommand{\eqd}{\stackrel{d}{=}} % equal in distribution/law/measure
\newcommand{\argmax}{\textrm{argmax}}
\newcommand{\argmin}{\textrm{argmin}}
\newcommand{\conv}{\textrm{conv}} % for denoting the convex hull

% Theorem-like declarations
\theoremstyle{plain}
\newtheorem{theorem}{Theorem}
\newtheorem{corollary}[theorem]{Corollary}
\newtheorem{lemma}[theorem]{Lemma}

\theoremstyle{definition}
\newtheorem{definition}[theorem]{Definition}
\newtheorem{example}[theorem]{Example}

\theoremstyle{remark}
\newtheorem{remark}[theorem]{Remark}



\setlength{\oddsidemargin}{0.25 in}
\setlength{\evensidemargin}{-0.25 in}
\setlength{\topmargin}{-0.6 in}
\setlength{\textwidth}{6.5 in}
\setlength{\textheight}{8.5 in}
\setlength{\headsep}{0.75 in}
\setlength{\parindent}{0 in}
\setlength{\parskip}{0.1 in}

\newcommand{\lecture}[4]{
   \pagestyle{myheadings}
   \thispagestyle{plain}
   \newpage
   \setcounter{page}{1}
   \noindent
   \begin{center}
   \framebox{
      \vbox{\vspace{2mm}
    \hbox to 6.28in { {\bf Stat210A:~Theoretical Statistics \hfill Lecture Date: #4} }
       \vspace{6mm}
       \hbox to 6.28in { {\Large \hfill #1  \hfill}  }
       \vspace{6mm}
       \hbox to 6.28in { {\it Lecturer: #2 \hfill Scribe: #3} }
      \vspace{2mm}}
   }
   \end{center}
   \markboth{#1}{#1}
   \vspace*{4mm}
}

% Local Macros Put your favorite macros here that don't appear in
% stat-macros.tex.  We can eventually incorporate them into
% stat-macros.tex if they're of general use.

\begin{document}

\lecture{Probabilities R Us}{Michael I. Jordan}{Winnie Pooh}{August XXX, 2014}

\section{General Suggestions}
\begin{itemize}
\item Use the macros in \verb+stat-macros.tex+ when possible.
  Also feel free to use our own favorite macros, and send them to
  us if you feel that they'll be generally useful.
\item To make these scribe notes maximally useful, you might consider
  adding textbook references for theorems, definitions, examples, etc.
  Make use of \verb+stat.bib+.
\item Add \verb+\label+'s for important theorems, definitions, and
  examples.  This will facilitate future cross-referencing.  
\item If you'd like to refer back to a theorem we've proved in an
  earlier lecture, the easiest thing would be to refer to the
  equivalent theorem in one of our textbooks.
\end{itemize}
\section{Some Sample Stuff}
\begin{definition}[Covering Number]\label{def:covering-number}
  \citep[p.~25]{pollard} Let $Q$ be a probability measure on $S$ and
  $\cf$ be a class of functions in $\cl^1(Q)$. For each $\eps>0$
  define the covering number $N_1(\eps,Q,\cf)$ as the smallest value
  of $m$ for which there exist functions $g_1,\dots,g_m$ (not
  necessarily in $\cf$) such that $\min_j Q|f-g_j| \le \eps$ for each
  $f$ in $\cf$. For definiteness set $N_1(\eps,Q,\cf)=\infty$ if no
  such $m$ exists.
\end{definition}

\begin{remark} Although Pollard doesn't define it, the log of the
  covering number (see Definition~\ref{def:covering-number}) is
  called the metric entropy.
\end{remark}

\begin{lemma}[Basic Inequality for Convex Classes]\label{lemma:basic-ineq-convex-class}
  \citep[Lemma 4.5, p.~51]{geer} Suppose $\cp$ is convex.  Then
  $$
  h^2(\hat{p}_n,p_0) \le \int \frac{2\hat{p}_n}{\hat{p}_n+p_0}\,d(P_n-P).
  $$
\end{lemma}
\begin{proof}
  We first prove step 1, then use induction to complete steps 2
  through n, and then we're done.
  \begin{enumerate} 
  \item Show that $A<B$

    To show $A<B$, use a property of the MLE $\hat{p}_n$ and the
    convexity of the class $\cp$.

  \item Show $B<C$
    
    By concavity of $\log$, we know $\log(x) \le x-a$ for $x>0$.  The inequality follows from this.
    
  \item Proof follows by induction

    This part's obvious.
  \end{enumerate}
  
\end{proof}

The following theorem follows from
Lemma~\ref{lemma:basic-ineq-convex-class}. It is from \citet[Thm 4.6, p.~51]{geer}:

\begin{theorem}[Hellinger consistency of MLE for convex classes]
\label{thm:mle-convex-class}
  Suppose $\cp$ is convex.  Assume moreover that
  $$
  \frac{1}{n} H_1(\delta,\cg^{(\conv)},P_n) \convp 0, \quad \mbox{for all } \delta>0
  $$
  Then $h(\hat{p}_n,p_0) \convas 0$.
\end{theorem}
\begin{proof}
  Follows from some other stuff.
\end{proof}


\subsection{A subsection heading}

Here is how to typeset an array of equations.

\begin{eqnarray}
	x & = & y + z  \label{eq:notInteresting} \\
%
     \alpha & = & \frac{\beta}{\gamma} \label{eq:interesting}
\end{eqnarray}

Notice that equation~(\ref{eq:interesting}) is interesting, while
equation~(\ref{eq:notInteresting}) isn't really.

And a table.
                                                                                
\begin{table}[h]
\centerline{
    \begin{tabular}{|c|cc|}
        \hline
        \textbf{Method} & Cost & Iterations \\
        \hline
        Naive descent       & 12 & 200 \\
        Newton's method & 500 & 30 \\
        \hline
    \end{tabular}}
\caption{Comparison of different methods.}
\end{table}

\bibliographystyle{apalike}
\bibliography{stat}

\end{document}



