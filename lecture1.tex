\documentclass[twoside]{article}
\usepackage{amsmath,amssymb,amsthm,graphicx}
\usepackage{epsfig}
\usepackage[authoryear]{natbib}

\newcommand{\reals}{\mathbf{R}}
\newcommand{\integers}{\mathbf{Z}}
\newcommand{\naturals}{\mathbf{N}}
\newcommand{\rationals}{\mathbf{Q}}

% Caligraphic alphabet
\newcommand{\calr}{\mathcal{R}} % only because \cr already taken
\newcommand{\ca}{\mathcal{A}} \newcommand{\cb}{\mathcal{B}} \newcommand{\cc}{\mathcal{C}} \newcommand{\cd}{\mathcal{D}} \newcommand{\ce}{\mathcal{E}} \newcommand{\cf}{\mathcal{F}} \newcommand{\cg}{\mathcal{G}} \newcommand{\ch}{\mathcal{H}} \newcommand{\ci}{\mathcal{I}} \newcommand{\cj}{\mathcal{J}} \newcommand{\ck}{\mathcal{K}} \newcommand{\cl}{\mathcal{L}} \newcommand{\cm}{\mathcal{M}} \newcommand{\cn}{\mathcal{N}} \newcommand{\co}{\mathcal{O}} \newcommand{\cp}{\mathcal{P}} \newcommand{\cq}{\mathcal{Q}} \newcommand{\cs}{\mathcal{S}} \newcommand{\ct}{\mathcal{T}} \newcommand{\cu}{\mathcal{U}} \newcommand{\cv}{\mathcal{V}} \newcommand{\cw}{\mathcal{W}} \newcommand{\cx}{\mathcal{X}} \newcommand{\cy}{\mathcal{Y}} \newcommand{\cz}{\mathcal{Z}}

\newcommand{\ind}[1]{1_{#1}} % Indicator function
\newcommand{\pr}{P} % Generic probability
\newcommand{\ex}{E} % Generic expectation
\newcommand{\var}{\textrm{Var}}
\newcommand{\cov}{\textrm{Cov}}
\newcommand{\sgn}{\textrm{sgn}}
\newcommand{\sign}{\textrm{sign}}
\newcommand{\kl}{\textrm{KL}} 

\newcommand{\law}{\mathcal{L}}  % \law{X}, the measure associated with r.v. X
\newcommand{\normal}{N} % for normal distribution (can probably skip this)
\newcommand{\eps}{\varepsilon}

% Convergence
\newcommand{\convd}{\stackrel{d}{\longrightarrow}} % convergence in distribution/law/measure
\newcommand{\convp}{\stackrel{P}{\longrightarrow}} % convergence in probability
\newcommand{\convas}{\stackrel{\textrm{a.s.}}{\longrightarrow}} % convergence almost surely

\newcommand{\eqd}{\stackrel{d}{=}} % equal in distribution/law/measure
\newcommand{\argmax}{\textrm{argmax}}
\newcommand{\argmin}{\textrm{argmin}}
\newcommand{\conv}{\textrm{conv}} % for denoting the convex hull

% Theorem-like declarations
\theoremstyle{plain}
\newtheorem{theorem}{Theorem}
\newtheorem{corollary}[theorem]{Corollary}
\newtheorem{lemma}[theorem]{Lemma}

\theoremstyle{definition}
\newtheorem{definition}[theorem]{Definition}
\newtheorem{example}[theorem]{Example}

\theoremstyle{remark}
\newtheorem{remark}[theorem]{Remark}



\setlength{\oddsidemargin}{0.25 in}
\setlength{\evensidemargin}{-0.25 in}
\setlength{\topmargin}{-0.6 in}
\setlength{\textwidth}{6.5 in}
\setlength{\textheight}{8.5 in}
\setlength{\headsep}{0.75 in}
\setlength{\parindent}{0 in}
\setlength{\parskip}{0.1 in}

\newcommand{\lecture}[4]{
   \pagestyle{myheadings}
   \thispagestyle{plain}
   \newpage
   \setcounter{page}{1}
   \noindent
   \begin{center}
   \framebox{
      \vbox{\vspace{2mm}
    \hbox to 6.28in { {\bf Stat210A:~Theoretical Statistics \hfill Lecture Date: #4} }
       \vspace{6mm}
       \hbox to 6.28in { {\Large \hfill #1  \hfill}  }
       \vspace{6mm}
       \hbox to 6.28in { {\it Lecturer: #2 \hfill Scribe: #3} }
      \vspace{2mm}}
   }
   \end{center}
   \markboth{#1}{#1}
   \vspace*{4mm}
}

% Local Macros Put your favorite macros here that don't appear in
% stat-macros.tex.  We can eventually incorporate them into
% stat-macros.tex if they're of general use.

\begin{document}

\lecture{Lecture 1: Statistical Decision Theory}{Michael I. Jordan}{K. Jarrod
Millman}{August 28, 2014}

\section{Notation}

data $X=(X_1,X_2,...,X_n)$ for $n \in \{1,2,...\}$

assumed that the distribution of $X$ belongs to a set $\mathcal{P} = \{P_\theta
: \theta \in \Omega\}$ of probability distributions.  We call $\mathcal{P}$ the
model, $\Omega$ the parameter space (i.e., the set of all possible values of
$\theta$). 

A statistic $\delta(X)$ is a function of the data $X$ which is may be real- or
funtion-valued.

In statistical estimation the goal is to find a statistic $\delta$ such that
$\delta(X)$ is "close to" $g(\theta)$, where $\theta$ is the true parameter
that indexes $X$'s distribution:
\begin{equation}
X \sim P_\theta.
\end{equation}
We say that $\delta$ is an estimator of $g(\theta)$.  To make clear what we mean
by "close to" above we need to turn to Decision Theory.

\section{Decision Theory Framework}

\begin{figure}[ht]
\begin{minipage}[b]{0.45\linewidth}
\centering
\includegraphics[width=\textwidth]{../fig/risk1-crop.pdf}
\caption{default}
\label{fig:figure1}
\end{minipage}
\hspace{0.5cm}
\begin{minipage}[b]{0.45\linewidth}
\centering
\includegraphics[width=\textwidth]{../fig/risk2-crop.pdf}
\caption{default}
\label{fig:figure2}
\end{minipage}
\end{figure}

\section{Example}

\begin{figure}[ht]
\centering
\includegraphics[width=0.4\textwidth]{../fig/risk3-crop.pdf}
\caption{default}
\label{fig:figure3}
\end{figure}

\bibliographystyle{apalike}
\bibliography{stat}

\end{document}
